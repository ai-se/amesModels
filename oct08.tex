\documentclass[12pt]{article}
\usepackage{fullpage} 
\usepackage[margin=1in]{geometry}
\usepackage{times}
\usepackage{url}

\addtolength{\itemsep}{-0.05in}
%\usepackage[small,compact]{titlesec}
%\usepackage[small,it]{caption}
 
\newcommand{\bi}{\begin{itemize}[leftmargin=0.4cm]}
\newcommand{\ei}{\end{itemize}}
\newcommand{\be}{\begin{enumerate}[leftmargin=0.4cm]}
\newcommand{\ee}{\end{enumerate}}

 
\newcommand{\tion}[1]{\S\ref{sect:#1}}
\newcommand{\fig}[1]{Figure~\ref{fig:#1}}
\newcommand{\eq}[1]{Equation~\ref{eq:#1}}
\usepackage[shortlabels]{enumitem}  
\usepackage{cite} 
\def\baselinestretch{0.92}
\setlist{nosep}

 \usepackage[font={small}]{caption, subfig}
\setlength{\abovecaptionskip}{1ex}
 \setlength{\belowcaptionskip}{1ex}
 \setlength{\floatsep}{1ex}
 \setlength{\textfloatsep}{1ex}
\usepackage[compact,small]{titlesec}
 
\begin{document} 
\begin{center}
{\Large {\bf  Verifying Safety of NextGen Models: A Rational Approach}}\\
Tim Menzies (PI), Computer Science, NC State, USA tim.menzies@gmail.com\\ 
Neha Rungta,  SGT Inc/NASA Ames,  neharungta@gmail.com
\end{center}

\underline{{\bf ABSTRACT:}}  
%We seek safer systems. NASA  uses model-based methods to find ways to improve safety in complex systems. 
The current National Airspace (NAS) is quite safe and accidents are at a record low. Over the past few years, the US has embarked in a transformation of the Air Transportation System (ATS) to address the expected increase of air traffic in the US. The goal is to increase efficiency without compromising safety. There is a critical need for new safety evaluation techniques to verify and validate the interactions of humans and autonomy in the context of new complex software systems and changing roles of the human operator. 
%But can we trust the conclusions achieved via these models? Some of these models are so large that they defeat formal verification tools. 
The current state of the art to assess interactions of humans and autonomy is to model the air transportation system as a distributed, interactive system of systems with authority and autonomy assigned to both humans and automation at multiple levels. The scale and complexity of these models make the use of traditional verification and validation techniques intractable. 

Hence, we  propose   {\bf exciting and unexplored} research to develop a {\bf new concept} that combines
heuristic optimizers
and formal verification tools in order to better understand huge models. 
 In this approach,  optimizers quickly select the smaller, most interesting regions of the model.
 While previous work has explored these two approach separately, 
{\bf no 
previous work has studied their combination}. Creating this combination is {\bf much more than a mere tool
or process project}. Rather, it requires  {\bf fundamentally new ideas} for  utilizing a variety of signals
collected from complex systems.  
We propose to use the optimizers to find  guides
that can assist model-checking based verification when exploring these huge models. Our approach is based on the  {\em principle of rationality}~\cite{Newell88} which states that experienced operators of any system know that certain actions are most relevant to important outcomes. That is, within the very large decision space of huge models, there exists a  smaller space of relevant and rational actions. 
Our hypothesis (to be tested in this work) is in this smaller space, it is possible that formal verification can thoroughly explore a model.

\underline{{\bf  THE DOMAIN:}} Our work is {\bf based on a current NASA  aerospace architecture}.
NASA and the FAA agree that the US need to modernize the ATS and to implement NextGen (Next Generation Air Transportation System) to accommodate the traffic increase over the next 15 years; Europe is going through a similar effort with SESAR. NextGen needs to provide at least the same, if not a better, level of safety. The  NextGen IWP has a requirement (R-1440) that calls for new and improved V\&V techniques for complex systems. 

Recent work at NASA Ames models several scenarios in the Brahms multi-agent framework~\cite{clancey1998brahms,SierhuisPhD}. Some examples include, Brahms models for conditions that led to the Air France 447 accident~\cite{hunter:aamas13}, sequence of events that led to the \"{U}berlingen collision~\cite{Rungta:2013}, and performed a comparative evaluation of operator workload between a single pilot operations concept versus a traditional two pilot
operation~\cite{Stocker:2015}. More recently, Brahms models have been developed for arrivals and departures at the La Guardia airport based on the work on Departure Sensitive Arrival Scheduling (DSAS) concept developed by the Airspace Operations Lab (AOL) at NASA Ames~\cite{dsas}. Through simulations of these models the interactions between pilots, controllers, and automation several issues regarding safety (Air France 447 and  \"{U}berlingen crashes); performance (DSAS concept), and human operator workload (single pilot operation) have been discovered, characterized, and fixed. 
 
 
\underline{{\bf  THE PROBLEM:}} 
ACT-R, SOAR, and other architectures provide the ability to model actual low-level cognitive processes~\cite{anderson1997act,laird1987soar,lebiere2013cognitive,liu2009qn,lundinsimulating}. However,  such modeling languages cannot  describe complex systems and interactions at a higher-level of abstraction, or   the decision making process of the human~\cite{pritchett2011simulating}. 
Other modeling tools, like the Brahms tool used at NASA Ames~\cite{Rungta:2013,Clancey:1998:BSP:306180.306196,brahms-semantics,brahms-jelia,SierhuisPhD} can build more extensive models that know about higher-level human cognitive reasoning. Hence, they are the tool of choice at Ames for modeling air traffic management (ATM), NAS operations, and even manned space flight scearios.

The problem with models is that even at a higher-level of abstraction, they can contain fairly extensive interactions as well as a large number of rational agents including their beliefs, goals, and actions. Hence, when verifying them, it is often required to severely constrain the analysis by either creating simpler models or bounding the scope of the verification.
%For example, recent work restricted the human interactions and activity in a single sector, and the air traffic controller actions to a few  rigidly pre-defined actions~\cite{MIDAS}. 
There is a pressing need to scale automated verification tools to meet the demands of NextGen. Decades of research at NASA has lead to rigorous and formal verification methods for model-based reasoning-- all of which suffer from problems of scale.  
 
\underline{{\bf  OUR PROPOSED SOLUTION:}} 
To make formal verification tools practical for large models,
we use  the   principle of rationality to find the relevant parts of a model.
Our proposal is to use the  GALE {\em  heuristic optimizers}~\cite{krall2015gale} to  ``run ahead'' of the formal verification tools to ``shout back'' the most important parts of the model to explore.   
GALE mimics the actions of a smart human operator of a model that finds the most useful regions within a model.
GALE
is an innovative combination of data mining and 
optimization algorithms:
\bi
\item
The data mining divides the data into tiny regions
with {\em best} and {\em worst} ends.
\item
The optimizer builds new solutions by mutating  the solutions
in a region towards the {\em best} end.
\ei
GALE inputs a random selection of possible model inputs. These
 inputs are clustered to   reduce  $n$ inputs into $\sqrt{n}$ clusters.
 These are then sub-divided into a tree of clusters and sub-clusters and sub-sub-sub-clusters, etc. Next, GALE walks down that tree performing {\em binary culling};
 i.e. prune away sub-trees that contain the undesired model outputs (for the NASA models,
 undesired outputs include: conditions that lead to safety violations; or dangerous spikes in operator workload; or undesired large deviations in performance; e.g. flight delays).

\newcommand{\KK}[1] {\em {\sffamily #1}}
When GALE finishes, it returns just the remaining tiny
regions   containing the most promising inputs.
From these regions, we can extract four sets of  input decisions that:
\bi
\item  {\KK Never} appear in the promising regions;
\item Appear at the {\KK Same} frequency at the {\em best}and {\em worst} ends;
\item Are {\KK More} frequent at the best end than the worst end;
\item Are {\KK Less} frequent at the best  end than the rowst end.
\ei
These tiny ranges, the above four sets, could be used
to enable better model verification using automatic verification tools. Firstly, the restrictions on the model
space mean that the verification tools need to only search
a small percentage of the total model space. Secondly,
the above four sets can be used to heuristically order the internal execution of the verification.

In this work we propose to use the Java Pathfinder (JPF)~\cite{jpf} framework to perform GALE-directed verification of Brahms models. JPF is an extensible Java
Virtual Machine (JVM) for developing and exploring
different verification techniques and application
domains. In previous work we have developed an extension to JPF that allows us to systematically explore all possible behaviors of Brahms models~\cite{Rungta:13}. JPF uses a generic model of the program state space consisting of {\em States}, {\em Choices} and {\em Transitions}.  {\em States} are restorable snapshots of a program execution along a particular path.  We leverage the choice generation mechanism within JPF to interact with the Brahms simulator and generate all possible choices at points of non-determinism within a Brahms model. This is an important advantage afforded by JPF, most model checkers do not provide user-configurable options to manage domain-specific types of non-determinism.

In this work we propose to use the output generated by GALE to direct the exploration within JPF. Specifically, we propose use constraints generated by GALE to perform a targeted exploration of the space using JPF. We will develop techniques that allow us to ensure:
\bi
\item Any exploration that reaches a {\KK Never} can be instantly curtailed;
\item Similarly, any exploration that changes a {\KK Some} decision
can also be instantly curtailed;
\item
The remaining bounded explorations can be sorted according to
how much they use {\KK More} and how little they use {\KK Less}.
\ei
Once sorted, the verification tools can then be unleased
to explore deeper into the model but only
on this smaller, most relevant portion of
the model.
\ei


\underline{{\bf  TECHNICAL DETAILS:}} 
The goal is to use the output of GALE to battle the state space explosion problem when using JPF for the verification of Brahms models. When JPF analyses a Brahms model along with its semantics, a new state is generated when the search within JPF encounters a point of non-determinism that has not yet been explored.  There are three main points of
non-determinism in the Brahms semantics  (1) probabilistic updates to facts and beliefs of agents, (2) a range of activity durations, and (3) picking the next task to perform. In large models this leads even with a controlled exploration of choices, the state space explosion is a problem. With GALE we hope to overcome this by exploring relevant choices rather a brute force exploration. We will use constraints from GALE to select states (the beliefs and facts of agents and objects) and choices (relevant decisions). This we believe will allow us to mitigate the state space explosion problem and yet be able to characterize the safety, efficiency, and other properties of the model. 

We recommend GALE for this process since it runs very fast.
In prior experiments with large NASA models~\cite{me15z}, GALE  found optimizations comparable to more standard methods, but hundreds of times faster; i.e. four minutes, not 7.5 hours.


To achieve its fast runtimes,
GALE uses  a top-down  {\em binary spectral clustering} algorithm~\cite{kamvar03} 
that recursively divides
the state space along their eigenvectors.
Standard top-down spectral learners require  $O(n^2)$ calculations at each level 
of the recursion to find the eigenvectors~\cite{boley98}. A faster method, called  
uses Faloutsos'  heuristic~\cite{Faloutsos1995} to
approximate eigenvectors   in $O(n)$ time~\cite{platt05}.
GALE uses these eigenvectors for a top-down binary clusterer. 
At each level, GALE  projects all  vectors onto a line connecting two distant vectors called $P,Q$ (generating this line takes $O(n)$ time\footnote{Faloutsos’ heuristic: Pick any vector $O$ at random. Let one pole $P$ by the
vector farthest from $O$. Let the other pole $Q$ be the vector farthest from $P$. Note that
finding $P,Q$ requires only $2n$ distance calculations across $n$ vectors.}).
The line is then divided in half
and GALE recurses on each half. 
%To  find  small clusters of size, say, $m=\sqrt{n}$,
%WHERE   build a binary tree of height $log_2\sqrt{n}$ with 
%$2^{log_2\sqrt{n}}=\sqrt{n}$ nodes. Each node   evaluates only two vectors $P,Q$; i.e.
%WHERE requires at most $2\sqrt{n}$ evaluations
GALE is very fast since it uses the above clustering to find and cull much of the  search space. When dividing the data and  evaluating   $P,Q$, if $P$ produces better model performance scores than $Q$, then GALE  culls   the worst half of the data (this approach cuts back $n$ examples to $\sqrt{n}$ using just $2*log_2{\sqrt{n}}$ evaluations).  For the final leaf cluster, GALE says that the {\em direction of improvement} us towards the pole that generates better performance scores. GALE can then build $n$ new vectors by mutate all the vectors in that leaf along the direction of improvement. These mutants become generation $i+1$ that can be clustered  and then mutated again to form generation $i+2$, etc. Note that  GALE make no assumptions of continuous or discrete-space vectors-- it runs on either using a distance predicate that can comment on the the similarity of pairs of input states. 

 
\underline{{\bf  SUCCESS CRITERIA:}}  To test
the value of this approach, we only
need to run the verification tools with and without
GALE. This work would be a success if, for very large models,
GALE+verification finds errors missed by just formal verification.
 
\underline{{\bf  WHY THIS TEAM?}} 
Our proposed approach is {\bf credible and reasonable} as it is based on mature research tools (JPF and GALE)
being used by researchers with extensive experience in these specific tools, and with NASA systems.
Dr. Rungta is an internationally recognized expert in
methods for applying automatic verification methods to models
~\cite{Rungta:2013,hunter:aamas13,backes:2013,rungta2012change,pasareanu:ase10,rungta2009efficient}.
She is also skilled in ``mixed methods'' that combine multiple techniques (such as the GALE+JPF approach
proposed here)~\cite{puasuareanu2011symbolic}.
Also,
Dr. Menzies is one of the co-authors of the original GALE 
paper~\cite{krall2015gale}. He also supervises
a research lab of eight Ph.D. students,  exploring  GALE and its extensions.


Note that much of the case study material required for this
work is readily available. For the last
12 months, Dr. Rungta has been conducting extensive studies
with BRAHMS models. Hence, she is well aware of current
concerns at NASA.
 

\newpage
\bibliographystyle{plain}
\bibliography{refs}
\end{document}



scale to larger models than model checkers. That
said, given the size of the models now being studied at NASA, these tools still suffer
from runtime limitations.
\ei
The problem with all the above is   state space explosion: in theory,
$N$ variables each with $V$ different values has a total space of $V^N$. In practice,
this number can be so large that it cannot be rigorously explored. For example, a system 
with 300 boolean variables (N=300,V=2) has $2^300$ states; i.e. more states that stars in the observable universe.


A recent report~\cite{nrc:nextgen} on the Next Generation Air Transportation System (NexGen) within the NAS discusses the lack of models that accurately capture the decision making process and actions of the human operators. The report states, \emph{``One might like to use fast-time analytic models and simulations, but unfortunately there are few human-system analytic models that are very predictive, and they are also very context sensitive. There are very
few analytic models that are up to being very helpful for NextGen, other than for modeling basic vision and hearing.''} The report states that the lack of such human-analytic models will be a problem area for the Federal Aviation Administration (FAA) and NASA as they move toward to the modernization of the NAS in the NextGen. One of the biggest challenge in the development of fast-time simulation that captures human behavior and decision making process is the trade-off between (a) fidelity of the modeled system with respect to the real-world system versus (b) the considerations of computation efficiency and tractability of performing analysis on such models. In this project we propose developing analysis techniques that can scale to complex human operator models of the NAS and are predictive with respect to the safety and performance of the modeled system. 


\section{Related Workd}

In the area of civil aviation, measuring a human operator's mental
state presents a range of challenges to researchers despite various
proposals including subjective rating scales \cite{NASATLX88},
secondary task performance \cite{DualTaskMethodology91}, and
psychophysiological measures such as eye movements, heart rate, and
respiration \cite{MetricsWorkloadKramer91}).  Further, traditional
hu\-man\--in\--the\--lo\-op (HI\-TL) evaluations such as the NASA Task
Load Index (TLX) cannot adequately test performance across the
operational envelopes of complex work environments during the design
stage of the system development cycle.



\section{Predictive Human operator models} 

In this work we propose the use of MAS models as the basis of
characterizing the decision making process, actions, workload, skill,
interactions with automation, and situational awareness of human
operators.  MAS models offer an ideal design abstraction for systems
involving both humans and automation. These models provide the ability
for predictive reasoning about various safety conditions such as
expected behavior of the autonomy, situational awareness of humans,
workload of the human operators, and the amount of time taken from the
start to the end of a complete or partial operation or procedure.

Brahms is a simulation and development environment originally designed
to model the contextual situated activity behavior of groups of people
in a real world context~\cite{clancey1998brahms,SierhuisPhD}. In previous work, we created Brahms models for conditions that led to the Air France 447 accident~\cite{hunter:aamas13}, sequence of events that led to the \"{U}berlingen collision~\cite{Rungta:2013}, and performed a comparative evaluation of operator workload between a single pilot operations concept versus a traditional two pilot operation~\cite{Stocker:2015}. More recently, we have developed Brahms models for arrivals and departures at the La Guardia airport based on the work on Departure Sensitive Arrival Scheduling (DSAS) concept developed by the Airspace Operations Lab (AOL) at NASA Ames~\cite{dsas}. 

The DSAS concept provides the ability to maximize departure throughput at LGA without impacting the flow of the arrival traffic; it was part of a research effort to explore NextGen TBO solutions to problems in the New York metroplex. The concept was prototyped in an HITL performed in 2014 that consider operational procedures related to co-ordination, timing, TSS schedule, and other display features available to controllers and supervisors at Center, TRACON, and Tower. The HITL results demonstrate that the DSAS operations have the potential to increase departure throughput at LGA by nine aircraft per hour with insignificant impact on arrivals~\cite{dsas}.  It was our goal to replicate the results of the study within the agent based framework of Brahms. Each scenario evaluated in the HILT of the DSAS study consists of approximately 1.5 hours real time of simulation. During this time, there are between 130 and 150 airplanes being managed by four enroute controllers, three TRACON controllers, and one tower controller at LGA who is responsible for departures and arrivals. The planes are landing at approximately 36 to 40 planes an hour. 

\subsection{Analysis of MAS models} 
Due to the various decision choices in the model even a single simulation run can take several minutes. This makes any form of exhaustive, proof derivation based techniques based on model checking, theorem-proving or other compraable approaches infeasible to apply~\cite{hunter:aamas13,raimondi:aamas14}. The goal of this work is develop techniques that allow us to successfully analyze the models human system interaction models such that they can be used to predict stuff about the modeled system. 

NASA is struggling to reason about very big, very complex models. NEHA! 

If we can reason about these models then NEHA! But if the complexity of those models
defeat us then NEHA!

Over the last twenty years, various
NASA research teams have tried solving this problem using a variety of techniques:
\bi
\item Theorem provers are a promising technology but require complex models to be augmented with extensive annotations describing their properties. Worse, the search space within theorems relating to real-world models is so large, that even state-of-the-art theorem provers cannot handle very large models.
\item Model checkers can scale to larger models. Such model checkers can test if a small number of most critical assertions can be violated. Like theorem provers, model checkers
have issues scaling up to large models.
\item Runtime verification tools can scale to larger models than model checkers. That
said, given the size of the models now being studied at NASA, these tools still suffer
from runtime limitations.
\ei
The problem with all the above is   state space explosion: in theory,
$N$ variables each with $V$ different values has a total space of $V^N$. In practice,
this number can be so large that it cannot be rigorously explored. For example, a system 
with 300 boolean variables (N=300,V=2) has $2^300$ states; i.e. more states that stars in the observable universe.

\section{Our Solution}



\noindent{{\bf GALE and Verification:}}

